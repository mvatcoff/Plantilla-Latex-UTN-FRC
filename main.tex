\documentclass{article}
\usepackage[utf8]{inputenc}
\usepackage{geometry}
\usepackage{graphicx}
\usepackage{indentfirst}
\usepackage[spanish]{babel}
\usepackage{amsmath}
\usepackage{textcomp}       % símbolo +- \pm -+ \mp
\usepackage{subfig}
\usepackage{pdfpages}
\usepackage{multicol}
\usepackage{color}
\definecolor{verde}{rgb}{0,0.6,0}
\usepackage{array}
\usepackage{multirow}

%Esto es para poner figuras en formato de 2 columnas
%\newenvironment{Figura}
%  {\par\medskip\noindent\minipage{\linewidth}}
%  {\endminipage\par\medskip}
  
\geometry{
 a4paper,
 left=15mm,
 right=10mm,
 top=20mm,
 bottom=20mm,
 }
\begin{document}
	
	\begin{titlepage}
		\centering
		\vspace{5cm}
		\begin{figure}[h]
			\centering
			\includegraphics[width=2.8cm]{Imagenes/utn_logo.pdf}
		\end{figure}

		{\scshape\LARGE Universidad Tecnológica Nacional \par}
		{\scshape\Large Facultad Regional Córdoba\par}
		\vspace{3cm}
		
		\begin{tabular}{c}
			\LARGE \textbf{Trabajo Practico N$^o \ X$}\\
			\LARGE \textbf{Aaaaa Bbbbb Ccccc Ddddd}\\
			\hline
			\textit{Cátedra de Xxxxx, Departamento de Ingeniería Electrónica}\\
		\end{tabular}
		
		\vspace{7cm}
		\begin{tabular}{rll}
            \Large \textbf{Profesor:} & \Large \\
            						  & \Large 
            
            \vspace{0.5cm}\\
            
           \Large \textbf{Estudiantes:} 
           & \Large Vatcoff, Mariano & \Large 76.771\\
           & \Large Aaaaaa, Bbbbbb & \Large XXXX\\
           \vspace{0.5cm}\\
            
            \Large \textbf{Curso:} & \Large 5R1 \\
            \vspace{0.5cm}\\
            \Large \textbf{Fecha:} & \Large \today  \\   
        \end{tabular}
\end{titlepage}
    
    \tableofcontents
    \thispagestyle{empty}
    \setcounter{page}{1}
    \newpage

%Con \include las secciones se ponen en paginas distinas, con \input van contiguas

%    \begin{multicols}{2}
        \include{Secciones/Introducción}
        \include{Secciones/Marco_Teorico}
        \section{Sección 1}

Esto es una referencia a la figura \ref{fig:troll_face}

\begin{figure}[!h]
	\centering
	\includegraphics[width = 0.8\textwidth]{Imagenes/S1/troll.png}
	\caption{Troll Face}
	\label{fig:troll_face}
\end{figure}

Esto es una referencia a la bibliografía \cite{Reliability-NASA}
        \section{Sección 2}



        \include{Secciones/S3}
        \section{Sección 4}



        \include{Secciones/S5}
        \include{Secciones/Conclusiones}
%    \end{multicols}
	
	\nocite{*}
    \bibliographystyle{plain}
    \bibliography{references}
\end{document}
